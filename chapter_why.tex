\chapter{为什么要与人妻交往}

\section{与人妻交往的经济学分析}

很多人把女人比作汽车。有五菱宏光,有坦克300,有丰田花冠,有奔驰S级。对性取向正常的男性来说,对女人有需求是正常的,但是拥有女人的方式也多种多样。就像如果你必须要用汽车,你可以买新车、可以买二手车、可以租车、可以打车、可以拼车。而与人妻交往,就相当于你直接去隔壁老王家把他家的汽车开走了。

现在中国和所有西方国家的男女关系,用车来比喻,就是一个极其混乱而低效的汽车市场。首先是新车销售价越来越高。彩礼几十万,买房加名的话,可以到几百万上千万,酒席也要几万几十万。然后在这之前,还有长期的考验,考验过程都要花钱。

你以为你花了这些钱能买到一辆新车,结果人家给了你一个爆缸、爆胎过的二手车、事故车,做了板金,然后还按照新车的价格卖给你。如果你发现用新车价买了二手车了想要找人理论,他们还会倒打一耙,给你扣个``新车情结''的帽子,说什么二手车开起来和新车一样,还说我歧视二手车、事故车,真正优秀的赛车手不会嫌弃二手车等等让人听不懂的话。

更可怕的是,你花钱买完车,开了两个月,某天回家车没了。结果发现你的汽车自己启动了自动驾驶模式开走了。然后你的手机收到一张离婚协议书,叫你签名。这时候你只能去小红书发一篇笔记:``有没有靠谱的北京/上海/深圳离婚律师推荐?'',正文是``结婚两个月老婆拿走78万彩礼,找某知乎李姓大V在境外购入一个比特币转走了,然后要求分走北京星河湾/上海檀宫/深圳波托菲诺纯水岸6000万豪宅折现3000万现金。''

然后你想着,这种情况下,只有傻子才买车。所以你有出行需求就想着打车、租车。结果一打车,你就被拘留了。拘留的时候又说了一堆你听不懂的话,什么你打车是侵犯汽车权益、侵犯汽车尊严、物化汽车了。还说什么有一些车出售的时候说是自驾车,结果被司机强迫出来开网约车了。等等等等。

在这种情况下,其实从经济上说,对大部分人来说,最有性价比的,就是睡别人老婆。购车款彩礼别人出,房贷车贷别人出,车险医疗保险别人出,你只要负责去享受就行了。不小心搞大人妻的肚子,还有``铁绿帽法''\footnote{指在中国、法国等国家,丈夫有责任供养妻子婚内出轨所生子女的法律。让你戴的这个绿帽像铁帽子一样,永远摘不下来。}兜底,不用负责。而且一切完全合法,也符合现代价值观(见后文),何乐而不为呢?

而且,我们不仅要找人妻,还要找有钱人的老婆。有钱人的老婆一般都是我们一般无法负担得起的昂贵美女。但是当她们成为人妻之后,按照这本指南操作,可能不花钱直接享用在这个时代本来专属有钱人的美女。还有什么比睡有钱人老婆更快乐的事情呢?

\section{与人妻交往的法理分析}

首先,虽然我不是律师,但是我的法律能力绝对超过绝大多数律师。我曾经起诉美国总统,在美国联邦法院和美国联邦检察官辩论不占下风。而且更重要的是,我在这里说的绝对是真话实话。

记住,不要相信抖音等任何中国的平台的所有的``律师普法视频''。在中国,只有老老实实的读法条和判例才有意义。(虽然中国不是判例法国家,但是从判例的分布可以倒推法院对法条解读方式的概率分布。)中国大陆平台对普法视频的审核标准只有一个:维稳。就是只要能达到维稳的目的,律师胡言乱语的视频也是可以发出来的。而如果一个律师,仅仅讲述一个真实的判例,只要不利于``维稳'',也可能立刻被删。比如中国有很多女的结婚后不让碰,然后几个月甚至几天后就跑路离婚,然后彩礼一分不退的案例。但是你如果要在抖音讲这些案例,是发不出来的。所以如果你靠抖音而不是法条或者科学取样出来的案例,你是无法了解到真实的法律的。

\subsection{出轨的合法性}

\textbf{在中国和所有西方国家,人妻婚内出轨都是合法的。}出轨合法的核心是,女人有着完全的性自主权和生育自主权。无论你是女人的父母、男朋友、配偶、情人、子女、阿訇、校长、老板、党支部书记,你都没有权力强制干预任何一个女人的身体自由和生育自由。所以包办婚姻、丈夫强制要求妻子性爱、生育之类的行为,是100\%违法的行为。甚至可以判定为属于强奸罪、强奸未遂之类的严重暴力犯罪。

郭菊阳说服我跟她上床的时候,说过她结婚是被她爸胁迫的,她内心最爱的其实是我。那么如果她没有做伪证的话,按照中华人民共和国法律,她老公应该是犯了强奸罪,应当处以3年以上10年以下有期徒刑。然后她爸属于强奸罪的共犯,应当处以1年以上3年以下有期徒刑。如果郭菊阳是做伪证,那么应当处三年以下有期徒刑或者拘役。

同样的,\textbf{在中国和所有西方国家,黄毛和人妻约会都是合法的。}首先,一个妇女,是有完全的身体自主权的。也就是说,一个妇女的阴道和子宫,都不是任何人的所有物。不是父亲的,也不是配偶的,更不是网友的。所以黄毛和人妻约会,不侵犯任何人的任何法律保护的权益。反而黄毛给人妻提供了在婚内无法体验的快乐体验,给别人带来了额外的收益。

也就是,任何人都没有对人妻身体的占有权、支配权,但是人妻有着100\%追求自己幸福的权力,所以黄毛和人妻约会也是完全合法的。

前面说的,黄毛和人妻交往、约会是完全合法的。那么自然,苦主也是没有任何权力对人妻或者黄毛进行任何暴力行为。

黄毛与人妻的生命权和健康权,比人妻的性自主权还要高,更不用说苦主对人妻的幻想中不存在的控制权。苦主对黄毛和人妻并不存在对其他人不存在的施暴的权力。

虽然中国婚姻法规定,夫妻之间有忠实义务,但是这种义务仅仅是原则性的,即使违反了,也只是在离婚时多判一点财产。而这的优先级其实是属于最后的。最高人民法院的法律解释说明,任何人无法仅仅依靠这一条义务进行维权。

而各种判例也是一直遵循这种原则。各种离婚判例,如果苦主对出轨的人妻施暴,在离婚中,家庭暴力的重要程度远远大于所谓违反忠实义务,财产分割都会偏向于女方。

\subsection{性自由的历史根基}

前面说到,黄毛和人妻约会的行为,无论是对黄毛还是人妻,无论是在中国,美国还是其他任何西方国家,都是完全合法的。这个原因就是,身体和性爱自主权,是根植于法国大革命和美国独立战争,历经美国南北战争、巴黎公社、辛亥革命、俄国十月革命、二战反法西斯战争、中国人民的伟大的解放战争、还有美国民权运动,全球一波又一波的革命之后,确立下来的基本原则。

法国大革命的人权宣言写道``人生而自由''。1950年新中国第一部法律《婚姻法》发表,毛主席盛赞为``普遍性仅次于宪法的根本大法''。

也就是说,这种女性解放和自由,还有和人妻约会的合法性,是宪法水平、甚至是全世界所有文明国家的立国根基水平的。法律的保护是来源于最底层的。除非再经历一场二战水平烈度的战争,不可能有所改变。

\subsection{婚姻的法律起源}

首先,中国古代的婚姻制度,在现在中国可以说是从文化上、法律上、和道德上都完全断绝传承了,除了共用``婚姻''这个汉语词,和现在的婚姻没有关系。

现在无论是东亚还是西方的婚姻制度都是源自中世纪天主教的婚姻制度。中世纪天主教是一夫一妻制。天主教的婚姻是神圣的,婚姻必须基于爱情。这种爱情不只是现在的见色起意发情,而是一种神圣的感情。也就是夫妻双方对对方,都需要像爱上帝一样全身心的爱对方,夫妻双方都不能和配偶以外其他人有性行为,亵渎婚姻相当于渎神。然后在中世纪天主教婚姻制度下,离婚基本上都是不被允许的。天主教中,这种夫妻之间的爱情,是高于繁衍和后代的。甚至婚姻本身和繁衍后代无关,所以即使无子嗣也不能离婚。

到16世纪,英国国王亨利八世(就是那个娶了6个老婆砍了两个老婆的杀妻狂魔亨利八世)为了能够随意离婚换老婆,宣布脱离天主教会,创立英国国教会,让自己成为英国国教会的元首。自此英国的婚姻制度,就成为了一个可以随意离婚的天主教式婚姻。

而这种新教式婚姻制度,也就是这种临时补丁式的可离婚式天主教婚姻,随着西方文化的全球扩张,慢慢占领了全世界,包括中国。这种过渡性的婚姻制度,其实很可笑,就是你可以随便多次更换配偶,但是在两次更换的中间,要像爱上帝一样全身心地爱配偶——这本身就是违反理智和博弈论的一种可笑的制度。

而之前说的性自主权,出轨合法的观念,又随着法国大革命、十月革命、中国革命,作为一个高于婚姻的人权概念,被各国所接受。这样又进入了一种比较合理的制度,也就是无论是否结婚,性行为和生育都不能违反女性的自由。如此一来,自由主义的逻辑又完全自洽了。

\subsection{离婚和铁绿帽法}

但是性自主权,又是完全独立于婚姻之上的。所以各国的离婚法律,法律精神的基础,还是基于那一套可离婚的天主教一夫一妻制。而中国的婚姻和离婚制度,在天主教的基础上,又加上了一套计划生育的道德观。在计划生育的价值观内,生育越多越邪恶,生育越少越光荣。所以在现代中国,那种没有生育的爱情,婚姻是最伟大的。也就是说,最伟大、最政治正确的中国爱情故事,就是李松坚和凌菲菲的爱情故事了。一个亿万富豪爱上一个二婚带男孩的女人,然后为了爱情结合,亿万富豪把一套5亿的檀宫别墅送给了这个女人。之后这个二婚带男孩的女人又因为爱情,爱上了一个保安队长,然后勇敢的追求爱情,和富豪离婚,和保安队长私奔追求爱情去了。

理解现代中国和西方婚姻的天主教背景,也就很好理解中国和西方的离婚制度了。也就是说只要一结婚,妻子都是丈夫的神。不管多久离婚,不管有没有小孩,不管妻子有没有出轨,只要丈夫比妻子富有,只要丈夫收入比妻子高,都需要大出血。

在美国,有一种制度叫做alimony,可以翻译为离婚赡养费。类似中国的离婚抚养费。Alimony的出发点是,如果没离婚,那么丈夫的收入都是妻子的,为了保证公平,离婚后,高收入的丈夫必须向低收入的妻子支付收入的一部分,保证妻子不会因为丈夫主动离婚而降低生活水平。当然美国的法律制度还是很注重男女平等的,如果妻子收入高于丈夫,也是一样。

而生育,子女,血缘在天主教、基督新教的婚姻制度里面是无关紧要的。所以子女抚养费,大部分时候都低于离婚财产分割甚至alimony。而妻子婚内出轨生下的小孩,默认也是丈夫的儿子,丈夫也是有必要一直支付抚养费的。

在美国的法律体系下,虽然绿帽是默认,但是丈夫如果不想认婚内非生物学子女的话,可以提起亲子鉴定并声明脱离父子关系。但是在法国,男性无权脱离婚内妻子出轨所产下子女的父子关系,必须一直支付抚养费,这就是我称为的``铁绿帽法''————像满清的铁帽子王一样,永远也摘不下。

而中国也是实行铁绿帽法,只不过在铁绿帽法之上实行和稀泥法。婚内妻子出轨生下的非亲生子女,或者继子、继女,在离婚后,男性也是有抚养义务的。但是中国特色和稀泥下,如果男方非常坚定严正抗议,或者舆论反扑比较大,法院会通过离婚财产分割偏向等方式补偿男方。也就是说,任何一个进入婚姻的男人,最好掂量一下自己在法院的能量和对戴绿帽帮别人养小孩的接受程度,因为中国默认是实行铁绿帽法的。

\section{与人妻交往的道德分析}

其实通过前面的各种分析,其实与人妻交往的道德性也是显然的了。

首先,很大一部分单身男性,因为成本太高无法结婚甚至无法有性生活。他们有权力用合法的方式满足自己的生理需求。

另外,很多女人在婚姻生活中,其实是很痛苦的,他们因为害怕一些过时的、法律不支持的、不合时宜的传统,顶着自相矛盾的道德,被迫和一个不喜欢的男人在一起生活,而如果人妻能和黄毛交往,她们也能够很快乐。

而唯一似乎受损的苦主们,他们以为受损的权利,真的有任何支持吗?有任何法律还是道德,认为一个女的只要领证,就要称为配偶的附庸,丧失自由吗?

那么,我们和人妻交往约会,又有什么道德缺陷呢?

如果我们找的是有钱人的老婆,那就更没有道德缺陷了。在这个所谓自由市场经济的时代,权力和资本勾结,有钱人、老人垄断了所有的社会资源。年轻男性读书、工作、创造了科技、互联网、AI等所有财富,却被中老年男人和富二代们夺去了绝大部分的收益,踩在我们的身体上,占有了所有的女人。然后还反过头来跟我们说要继续奋斗。

美国皮尤研究中心(Pew Research Center)研究表明,现在美国35岁以下年轻人占有社会财富总量的比例,一直在刷新美国200多年历史的最低点。而中国、日本、欧洲也无不如此。在这种几百年来对年轻人最不公平的社会之中,通过伟大的爱情,对有钱人的老婆进行再分配,是再正义不过的事情了。

\section{中国现代的价值观}

上面的法理和道德分析,用的是中国和西方法律上实质的现代自由主义。自由主义即使实际落地可能会造成一些问题,但是至少是逻辑自洽的。中学生可以随便恋爱上床,人妻也能随便出轨。而儒家思想,其实逻辑也是自洽的。中学生15虚岁的七夕节出花园\footnote{潮汕少男少女的成人礼。}过后就可以结婚了。

但是在真实的自由主义法律体系下,现在中国台面上的所谓主流的政治正确的价值观,我称之为``保进步守''。也就是极端保守地拥护大约100年前的进步主义。这套进步主义,可以说是一套中华农民劣化版的普鲁士-纳粹总体战价值观。妇女解放反而是附带。也就是说,无论男女,必须放弃一切的感情、娱乐,甚至要放弃家庭生活和生育,而把一切都投入到``工业发展''和``中华民族的伟大复兴''。所以需要封禁索尼PS2游戏机,需要禁止早恋,需要禁止一切和高考无关的东西,需要计划生育。所有现实生活中无穷的苦,都是靠承诺的``幸福的彼岸''来支撑。你想想你高考前是不是有着无数的彼岸、天堂的承诺和地狱的恐吓,和世间所有邪教又有何区别?理论上,这种普鲁士-纳粹式的进步主义叙事需要通过一场总体战,获得超额的收益,否则一整代人就白白牺牲了——最后文化停滞、全民创伤集体爆发、人口雪崩民族消亡。

而在几十年前之后,随着所有的理想和信念消亡,普鲁士价值观继续堕落到全世界最低劣的拜金主义、拜物主义价值观。什么精神追求都是政治错误的,那么那些没有坚持的人,也就只剩崇拜金钱了。再结合普鲁士价值观,双重恶心:禁止早恋、禁止娱乐、计划生育不是为了伟大复兴,也不是为了对美/日/俄的复仇总体战,而是为了金钱。这种价值观也迅速让上上下下全体腐化。到现在为止,中国女人人均郭菊阳,也就是可以牺牲一切,十分努力地用身体换取金钱的``奋斗鸡''。而中国男人,则是人均相信``赢了会所嫩模''的``奋斗嫖客''了。

我有一个在北京微软工作的朋友,曾经在北京多次重复嫖宿一个湖南省出身的18岁妓女。当时正值房产泡沫时期,每一次需要5000人民币。有一天事后,他抱着湖南鸡,像所有的男人一样,开始劝鸡从良。他忍不住询问,她长得这么漂亮,怎么沦落到北京做鸡呢?她很认真的回答,她这么努力是为了在北京买房,为了有一个自己的家。她开始算起来:每一次接客,被中介抽水后可以赚3000多块,然后趁年轻漂亮一天努力多接几单,干到二十六七岁就能够在北京买到一套1000万左右的房子了。说完,我朋友为她努力做鸡买房的爱国行为感动万分\footnote{我这个朋友,因为看过我在2011-2012年写的关于土地财政和地方债的文章,所以就坚决不买房而一直租房。实在是``不爱国''。},决定再操她一次。经过这么一轮交心的交谈,他们两人,又干得更用力了……我的朋友沉浸在18岁湖南鸡诱人的肉体之中,而湖南鸡离在北京买房的梦想又近了一步。这就是美好的爱情吧。

我在2021年第一次在深圳把郭菊阳约出来的时候,她和我一见面,就开始炫耀她在波托菲诺纯水岸15期4栋的那套价值6000多万的豪宅了。当时我看着她那既熟悉而又陌生的脸,像在《了不起的盖茨比》里面一样,感觉her voice is full of money(她的声音充满了金钱)。她看我毫无反应,又强调了一遍她的房子价值6000多万。当时她那骄傲的表情,像是一个已经成功到达目标的湖南鸡的MVP结算画面,不对,像是一个至少3倍超额完成目标的湖南鸡的MVP结算画面。说着说着,她又开始骂附近白石洲城中村还是哪里的什么建设计划,会影响她的房产价格。因为我实在是不感兴趣,左耳进右耳出,也忘了具体她当时骂的是什么了。或许也是因为她鸡成这样,才让我和朋友一样想要劝鸡从良吧。

但是话说回来,价值观是价值观,法律是法律。真的有人按着头逼着你奋斗,逼着你拜金吗?在学校的时候泡泡妞,上班的时候摸摸鱼,有钱就到处旅游吃吃喝喝,上班时间泡泡家庭主妇,下课的时候就泡泡中学生,纽约香港住腻了,就我一样买张机票在东京买套房旅居\footnote{那些在北上广深买房的,随便卖一套就可以在大阪和东京(港区以外)住得很舒服了。},写写书,追求一下艺术理想。这样的生活,难道不是开心多了?
