\chapter{如何让人妻爱上你}

\section{泡妞是一门科学}

其实量化交易的核心就一句话:交易是一门科学。只要你相信科学,相信市场是可以(从概率上)预测的,然后用科学的实证的方法研究,你就懂量化了。无论你是做高频交易还是低频交易,无论你是用统计模型还是AI,无论你是用自动下单系统还是手动把房子挂出去卖掉都不重要,科学的交易者必然把信``缠论''或者``一劳永逸地把大A打到10000点''的前现代土鳖踩在泥土里揉成渣。

我每次用科学的方法论在市场上轻松赚钱的时候,感觉就像是那些拿着火枪,身着铁甲,以一敌百征服、屠杀印加帝国的那168名西班牙征服者。而历史上对印加帝国的征服,又何尝不是掌握了科学的一方对前现代的一方单方面的碾压与屠杀呢?

我看今天绝大多数中国人,接受了中学甚至大学教育,但是却从来没有真正掌握科学精神和科学方法论。(甚至很多反而是被学校灌输了很多错误观念。)一遇到教科书里没有出现过的问题,就求助于迷信和伪科学。

我的大姨,中山大学毕业的金中前校长李丽丽在2007年的时候拿了我妈几十万,重仓中石油血本无归。她的投资决策是基于某种信仰而不是科学。而我在的一个上海中学校友群里面,一个上海中学和上海交通大学毕业的上海妹妹,每天对着手机研究我的星座、命盘和八字。这种愚蠢的行为,让人感慨她受到的科学教育都扔黄浦江里面去了。

\textbf{要学习怎么泡妞,第一步,就是要认识到,泡妞是一门科学,女人的心理和行为是可预测的,甚至是非常容易预测的。}

你需要把社会尤其是女人给你灌输的``女人是神''这个信仰或者说思想钢印给去掉。前面说过,这个信仰甚至是源于西方的天主教,根本不是中国文化。但是现在,有一群人在拼尽全力维护``女人是神''这个概念,一旦你试图开始审视、研究女人,你就像《楚门的世界》(The Truman Show)里面的楚门一样,会受到周围所有人的攻击:会说你是不尊重女性、物化女性,等等等等。但是你如果信了这些女人和龟男的话,你就会像A股市场里面信``10000点''的韭菜们一样,血本无归,甚至爆仓跳楼。

\section{泡妞的科学原理}

这一节可能会涉及一些比较专业的知识,主要集中在博弈论(game theory)。我会尽可能保留专业术语(以便有兴趣的读者深入研究)的同时,用通俗易懂的语言解释,让非专业的普通读者也能理解理论。只有理解了科学的理论,才能够举一反三,在实操中随机应变,立于不败之地。

关于泡妞的科学的基础,可以总结为以下四个公理。

\subsection{李新野泡妞第一公理:感觉人公理}

\begin{axiom}\label{a1}
在自由状态下,女人绝大多数决策都是基于感觉,而感觉是基于本能,而本能是基于人类上百万年原始社会中两性博弈进化出来的纳什均衡策略\footnote{如果你没学过纳什均衡,可以把纳什均衡策略理解为(局部稳定)最优策略。}。
\end{axiom}

现代进化论告诉我们,包括人类在内的动物,在进化论的筛选下,一切的行为,都需要服务于生存与繁衍。但是这显然无法解释很多问题。比如为什么现在中国美国的沿海大城市集中了那么多单身女性,宁愿断子绝孙也不嫁人。这就回到了一个其实很容易观测到的事实:女人绝大部分的决策,都是基于自身的感觉。

那如何连接上这两个看似矛盾的理论呢?其实很简单,就是女人的感觉,其实就是上百万年中进化出来的最有利于女人生存繁衍的本能。最近5000年的父系传统文明的科技发展实在是太迅猛了,而人类的本能还未进化到适应现代高科技社会。而女人在失去了三从四德的规训之后,靠着感觉,重新跟随原始社会的最优策略在生活。

如果说,现代经济学是基于用博弈论分析``理性经济人''(economic man)\footnote{``理性经济人''也翻译为``理性人''或者``经济人''。指的是在经济学研究中,我们假设所有的人都是自私、无情感地追求自身经济利益最大化。虽然现实中金钱并不是人类唯一的行为动机,但是这种简单的假设可以简化数学模型,让博弈论分析变得可行且简单清晰。}。那么我们可以把现代社会的女人称之为``感觉人''。我们完全可以直接假设她们做的所有事情,都是为了让她们感觉更好。而把满足感觉(也就是原始社会的最佳策略)作为``效用''(utility)\footnote{``效用''是指博弈论里每个参与者(玩家,player)试图最大化的一个数值。在理性经济人假设下,效用等同于金钱。而在感觉人假设下,效用等同于感觉,也就是对原始社会最佳男女博弈策略的接近程度。},便可以开始用博弈论研究感觉人(女人)的行为了。

\subsection{李新野泡妞第二公理:一夫多妻公理}

\begin{axiom}\label{a2}
一夫多妻是人类历史的常态,女人慕强和向上择偶是纳什均衡策略。
\end{axiom}

首先,结合公理\ref{a1},很容易也推导出以下定理。

\begin{theorem}\label{t1}
女人慕强和向上择偶是天性。
\end{theorem}

这里说一夫多妻\footnote{指一个男人有多个女配偶,包含一夫一妻多妾。}是``常态'',并不是讲所有男人都有多个配偶,根据费希尔原理(Fisher's principle)\footnote{指根据自然选择理论,包括人类在内雌雄异型不大的物种,性别比会接近于1:1。否则数量少的性别有进化优势,进而恢复平衡。},人类性别比接近1:1,每个男人有多个配偶在数学上是不可能的。

这里的常态也不是说每个女人都要做妾。实际上,我翻看各种资料,无论是中国古代的史料,还是现在穆斯林国家和撒哈拉以南非洲的社会状况,即使一夫多妻完全合法,也仅仅有30\%左右的男人会有两个或以上的配偶。

但是,只要社会中常态存在一夫多妻的家庭,只要进入一夫多妻关系是女人的选择之一,那么这就会深刻改变女人的感觉和择偶策略。

我们做一个思想实验,如果人类有这么一个开关,如果和第二个异性发生性关系,就会立刻死掉。或者更温和点,人类只能和第一个发生性关系的人产生后代。那么经过千万年的进化,无论是男人还是女人,都会倾向于找一个优秀程度在同性的百分位和自己差不多异性,安心结合繁衍后代。但是这明显不符合现实。

而因为超过一百万年来,人类一夫多妻是常态,女性主要负责抚养后代。那么男性最佳的择偶策略自然是会倾向于多找几个配偶,即使配偶优秀程度差于自己,但是能增加后代数量。而女性,因为一生生育数量有限,且需要为抚养后代付出更多,所以女性的最佳择偶策略便是把有限的生育资源留给更优秀的男性。尽量剩下与优秀男性的共同后代。

所以这一切就解释得通了,因为一夫多妻是常态,所以在上百万年的进化下,男女的天性符合一夫多妻的纳什均衡策略:男人只要看到年轻漂亮忠诚的女人,内心都不会拒绝。但是要让一个女人跟一个没有她优秀的男人结合甚至繁衍后代,女人的内心是会极其痛苦的,觉得自己在这个一夫多妻的择偶博弈中失败了,基因被污染了,没法获得优秀的后代。

结合公理\ref{a1}和公理\ref{a2},可以得出以下推论。

\begin{corollary}\label{c1}
女人慕强,道德上等价于男人多偶。
\end{corollary}

女性慕强和男性多偶,本质上都是在现代社会释放原始社会的天性。但是现在很多华师妹竟然可以心安理得地觉得自己和清华男是配得上的,然后社会都去合理化这种慕强择偶。但是一个清华男说要有找小妾或者``细姨''\footnote{闽南语的小老婆},全社会都会去谴责,这是很匪夷所思的事情。

然后结合公理\ref{a1}和公理\ref{a2},还能得到下面这个很让人震惊的的定理。

\begin{theorem}\label{t2}
任何同时解放妇女和实行一夫一妻制的民族,必然消亡。
\end{theorem}

在上万年的一夫多妻下,女人进化出了慕强的本能。所以女人和(在择偶吸引力上)比自己弱甚至与自己相当的男性交配、繁衍这件事,是发自内心的厌恶和痛苦。但是因为一夫一妻制,她们看得上的男人,根本看不上她们,不可能和她们进入一夫一妻的关系。所以女人的决策基于自身感觉,为了避免这种下嫁或者平嫁的痛苦,女人便会宁愿选择不婚不育,也不与看不上的男性结婚、交配、生育。

所以解放妇女加上一夫一妻制到最后,会有大约一半的女人终身不婚不育。这个民族自然是要消亡的。

现在中国的结婚率,还是因为有长辈催婚撑上来的,但是总和生育率也小于1了,中国人会以指数速度绝种。而美国,基本没有保进步守的人,所以稍微好点。一方面,美国有部分人有守可保,可以退回原教旨基督教、正统犹太教的传统,不像中国,孔子和儒家和三从四德已经被完全打倒了。另一方面,美国在自由主义上,又正在慢慢进步到逻辑自洽的全面解放,所以非婚生子几乎完全不受歧视。也算是事实上部分开放了一夫多妻。现在美国大约一半的婴儿是非婚生子——这还不包括婚内女人和其他男人生下的小孩。靠这两点,美国现在的总和生育率显著超过中国,但是还是远小于更替生育率。

所以,如果你用科学的方法来看,无论是中国、美国、北欧还是日本,在解决少子化的问题上的路径都走错了。这些国家都试图用提高女性权利和福利,也就是贿赂女性的方式来提高生育率。但这注定是徒劳无功的,因为女性对平嫁和低嫁的厌恶和痛苦,远远超过国家能给的几千几万生育补贴能给女性带来的快乐。所以凌菲菲和许洁被迫和她们看不上的李松坚上床,只能吸干李松坚几亿几十亿财产报仇,甚至故意把我的姐姐弄残废。即使这样,她们也无法真正解气。\footnote{见附录收录文章\hyperref[3faces]{《我的父亲李松坚》}和\hyperref[sister]{《我的姐姐李新莹》。}}

所以,要提高生育率,只能一方面堵,也就是恢复传统道德,也就是按头教化逼迫女人必须为对她有恩的男人生育;另一方面疏,就是恢复一夫多妻制,让一部分女人慕强心理比较强而嫉妒心比较弱的女性,可以有做小妾的选择。

\subsection{李新野泡妞第三公理:女人无爱情公理}

\begin{axiom}\label{a3}
女人只对子女有爱,对男人只有为获取基因的迷恋和获取供养和保护的索取。
\end{axiom}

这里``爱''定义为无条件付出的感情。而如果把``爱情''定义为男女之间无条件付出的感情的话,那么很容易得到以下推论。

\begin{corollary}\label{c2}
女人没有爱情。
\end{corollary}

根据公理\ref{a1},女人的天性和感觉,其实是服务于原始社会的生存和繁衍的。所以很好理解女人只会对自己的子女有无条件付出的感情。

而女人对男人的感情,其实服务于女人对男人的两大需求。第一个需求是提供优质基因。所以女人在看到本能认为的基因优秀的男人的时候,会陷入一种疯狂的迷恋,想要立刻和对方交配、繁殖。

但是这种迷恋并不是爱情。女人只会想和你上床,并不一定想和你发展长期关系,也很难给你真正的钱。\footnote{如果你想要从女人身上拿钱,你需要先让女人迷恋,再让女人得不到你,这时候女人就可能打钱了。这就是牛郎店赚钱的套路。但是你一旦跟她上床了,她就不可能给你钱了。}而且,这种迷恋来的快,走得也快。也就是女人口中所谓的``上头''和``下头''。毕竟已经上床了,女人想要得到的种的得到了,没有必要继续在这个男人身上浪费时间了。

女人对男人的第二个需求,就是要求男人提供供养和保护的需求。当然根据公理\ref{a1},女人这个需求也是有对应的感觉的,而这个感觉就是女人口中的``安全感''。

很多男人不理解,女人其实是被诅咒的,她们无时无刻不处于缺乏安全感的焦虑之中。这种焦虑比男人缺乏性生活还要痛苦。之前人人网(校内网)网红李硕曾经将这种焦虑感称之为``阴茎缺失焦虑综合症'',也算是非常精辟而准确的称谓了。女人是永远无法独自承担自身命运的,也无法独立面对这个世界。女人需要不断地确认是否能绑定一个强大的男人,让他有意愿源源不断地提供供养和保护。

其实除了凌菲菲等极少数聪明的女人,绝大部分女人对男人的控制和吸血,主要并不是精确计算的结果,而是被``安全感缺失''痛苦所控制。也就是一旦男人没有顺从女人,没有给钱,没有仪式感,女人就会痛苦,没有安全感,然后开始作闹。而当你通过语言、行动、仪式感等方式缓解了女人的痛苦,女人便会停止作闹,甚至会给你性交等好处。

这两种感觉的对象,有人称之为``情人''和``供养者'',或者``基因提供者''和``供养者''。那么这两者,到底哪种是爱情呢?只能说见仁见智了。前者是疯狂的迷恋,后者是缓解了女人的焦虑之后,产生的依恋。但是其实说实话,这两者本质上都不是爱情。毕竟前者本质上只是要基因,后者本质上只是想要供养与庇护,最多在满足之后给予身体接触。都不是像基督教要求的教徒对上帝的那种无条件的爱与崇拜。

而男人其实是和女人不一样的。就是男人其实是有一种无条件利女的蜂巢思维(hive mind)。也就是说,因为女人是族群繁育下一代的主体,所以保护女人,即使不是自己的配偶的女人,也对帮助族群整体的繁衍有利。就像蜂群里面,放弃了个体繁衍的工蜂和雄蜂,牺牲自己,为了蜂后的繁衍付出一切。其实男人和蜂群里面的工蜂和雄峰一样,有为女人无条件牺牲的本能。这就是为什么会有泰坦尼克号一等舱男乘客生存率比三等舱女乘客还低——一个男人,即使努力到了买得起一等舱,在社会看来,生命的价值不如一个三等舱的底层女人。这种人类整体无意识地对男人生命的漠视,被称之为男性可弃置性(male disposability),而人类这种对女人的超额关注,被称为女本位主义(gynocentrism)。

而男人这种无条件牺牲自己造福女人的利女本能,也就是男性对女性的爱情。而女性的安全感缺失,装可爱,装可怜和作闹之所以有用,也就是靠触发身边男性的这种利女本能(爱情),完成资源的掠夺和转移。

因为女人无论是对前者迷恋,还是对后者的依恋,出发点都是自身当下的情绪。也就是说,即使身边的男人不是女人子女的父亲,女人也会继续释放不安全感,作闹,掠夺身边男性的资源。这个当下的情绪,让女人无论子女的血缘关系掠夺每个身边的男人,这也是在原始社会下博弈的最优解。又印证了公理\ref{a1}。

比如最近有很多人妻出轨小孩非亲生的案件,我很喜欢看这里面对人妻的采访,她们的回答其实非常有意思。有一个人妻说:``我就是出去放纵了一次,有必要这么生气吗?''还有一个人说:``我都跟你那么多年了,他都叫你爸爸那么多年了,你就这么抛弃我们你有没有良心。''很多人看到这些言论非常气愤,但是你看到这里,你就会觉得这些女人其实诚实得甚至有点可爱。对那个情人,那个女人的迷恋的感情来得快,去得也快,也成功的获得了他的基因。而对于她的丈夫,她还是把她当作长期缓解焦虑感和不安全感的对象,确实是把他当依靠对象和丈夫看的。所以她本能地对丈夫离开感到愤怒和不满。

\subsection{李新野泡妞第四公理:间接探测公理}

\begin{axiom}\label{a4}
女人对男人的迷恋感觉基于对男人(在原始社会)的价值的间接判断。
\end{axiom}

女人的直觉,其实是进化成一个``高价值男人探测器''或者说是``优质男人探测器''的。但是这种探测并不是直接评估价值,而是通过间接的线索去判断这个男人的价值。

过去两年,我经常拿着某个有100个比特币的钱包给女人们看,做为一种社会实验。那些女人们,就算读到985大学,看到那串余额数字和数字背后的哈希值,都是面无表情。那些比较有礼貌的或者比较有知识的,会``哦''一声回应我。而那些既没礼貌也没知识的呢,就会露出一种对油腻中年暴发户``嫖客''的那种厌恶表情。

而与之形成鲜明对比的是,有一天晚上,我在上海和一个江西九江的妹妹约会,我们在卢湾区的日月光中心吃了一顿美味的日式烤肉之后,带她到上海西郊的一栋独栋别墅约会。在这种浪漫氛围的烘托之中,一切的进展都是那么简单而顺理成章。

但是这种浪漫氛围其实花不了多少钱!靠着我钱包里那100个比特币,即使不算未来比特币的增值和上海房价的持续崩盘,我也能够把自己复制四份,分别约一个彭泽妹妹、一个浔阳妹妹、一个柴桑妹妹和一个永修妹妹,然后分别到卢湾日月光、徐汇日月光、陆家嘴国金中心商场和静安嘉里中心,再分别吃一顿日式烤肉、一顿韩国烤肉、一份美式烤牛排、一份巴西烤牛排,再分别带她们到浦东、浦西、苏(州河)南、苏北四套独栋别墅里面约会上床过夜。我可以这样持续三十年,直到我完全阳痿为止,然后这100个比特币还有剩!

但是女人就是这样子,她们不会为100个比特币对我产生迷恋,但是迷醉的烛光晚餐和灯火辉煌的大别墅,可以让她们阴道湿润并开始排卵。即使前者的价值是后者的数万倍。

但是某种意义上,如果女性进化出对100个比特币的发情本能,那20年前女人是不是会因为没有人有比特币而无法发情而绝种?20年后的女人,是不是会因为全世界除了20年后的我之外几乎没人有100个比特币而绝种?所以女人这种``廉价七成正确''的择偶本能,其实也就可以理解了。至于剩下的三成,属于必要的基因缺陷消耗品。

当然女人对男人价值的间接探测,不只是烛光晚餐和大别墅那么简单。在后面的``吸引力法则''一节中,会介绍女人各种各样的``价值探测器''并教大家如何黑进(hack)女人这套上万年没有更新过的择偶软件。

\section{AI时代的母猴子}

回想我自己的人生,我走的很多弯路,感受到的很多痛苦,都是因为错误地把现代女人当作和我一样的君子,或者我母亲一样的贤惠的女人。

在清华和MIT,我师从姚期智教授和Silvio Micali教授两位图灵奖得主学习、研究博弈论\footnote{笔者在清华和MIT的研究方向是机制设计(mechanism design),一个理论计算机和经济学、博弈论的交叉学科,旨在研究如何设计商业、政治等博弈机制以优化利润(profit)、社会福祉(social welfare)等不同效用。}。之后我再用我图灵奖水平的大脑,和我的专业知识回头审视现代女人,构建了这套理论,也就看开了:其实现代女性,不过是一群活在AI时代的雌性灵长类动物(母猴子)罢了。

在太平洋两岸的最顶级的科技和AI公司里面,汉人男性贡献了大约3/4的开发工作。而华人女性,却每天打拳,用最恶毒的语言诅咒汉人男性。但是当你看完之前的理论分析,用原始人的眼光去看那群国女,也就释然了。如果你用逻辑去分析她们的语言,自然是狗屁不通。但是你如果用``感觉''去分析,一切就豁然开朗了。女权分子说的所有话,其实都是要释放女性两个底层的情绪:厌恶普通男性、安全感缺失。

前者的典型代表就是乡下女拳头子杨笠的``男人那么普通又那么自信''和官方女拳头子张桂梅的``溪流''、``沟壑''、``草芥''、``懦夫''。网络上现代女性铺天盖的对催婚的反感也是一样。上百万年一夫多妻制的筛选下,让女人们忍不住幻想一个清华毕业、美国归国、白手起家、身家百亿、然后还敢指着她们的鼻子骂她们是母猴子的强者从天而降到虹桥或者白云把她们娶走,而对身边千千万普通男性充满生理性厌恶,即使这些男性客观条件比她们好得多。

而后者的典型,就是清华学姐诬陷案和追风小叶被诬陷案。女性在没有强大男性庇护的时候,怕被普通男性侵犯的天性已经强大到让她们极度焦虑、歇斯底里。

所以说实话,现代社会大龄剩女也是妇女解放和一夫一妻制的牺牲品,她们同时受无法找到优秀男性和害怕被普通男性骚扰的焦虑感折磨的痛苦,实在是太悲惨了。

在AI时代,AI软件工程师无疑是全世界拥有最强生产力的一批人,也是战斗力最强的一批人。在此时此刻,我们汉族男性无疑已经有能力开发并量产超廉价AI自走杀戮机器人,可以在几天内消灭一个国家的所有人口而不需要担心核冬天(nuclear winter)的到来。假设美国能给愿意移民美国的国男程序员一人发五个金发白妹,美国能坐稳世界第一强国500年。可惜没有一个国家这么有眼光。

回到正题:虽然国男尤其是程序员和工人,是全世界真实实力最强的一批人,但是美国没有给我们发五个金发白妹,而国女也不会对真实的实力产生感觉。所以,如果你要让人妻和未婚女性喜欢上我们,需要做的是在AI时代增强自身真实实力的同时,也要同时让自己在女人眼中是一个原始社会中的强者。

接下来的一节,我会具体讲述如何让自己看着像一个原始社会中的强者,进而吸引母猴子。但是因为一个人的时间精力是有限的,实际操作中,这两者经常是矛盾的。至于多少时间用于增强自身真实实力,多少时间用于装原始社会的强者来陪母猴子玩,交由读者自行判断。

\section{吸引力法则}

\subsection{永远保持多偶}

\textbf{女人天性永远无法理解为什么一个优秀的男人会保持单偶。}

一夫一妻制,在西欧仅仅实行了一千多年,而在包括中国在内的全世界所有其他地方都没有根基。也就是说,女性根本没有进化出一夫一妻对应的特质。所以在原始社会中,一个男人如果优秀,自然会吸引很多女人,那么这个男人自然会选择同时有几个配偶。于是女人有一个天性,如果看到一个男人有很多或者很优秀的配偶(在现代就是女朋友或者老婆),这个人大概率是一个(在原始社会)优秀的男人,这个女性就很容易会不由自主地产生对这个男性的迷恋,想要也跟这个男性交配。女性的这种心理,有人称之为预选机制(preselection)。

当然,一个多偶的男人,和给女人提供安全感,是矛盾的。所以其实也是一个双刃剑。之前我曾经做过大样本的对照试验,就是对一部分女人直接说我要多偶,对一部分女人不提想要多偶。结论其实很有趣,就是直接跟女人说要多偶的那个样本,对我快速上头的更多。尤其是我说我要多偶之后,她们往往会突然觉得我这个人很有趣,也很有种,然后就开始热情聊天,快速上头。但是负面是,当跟她们上床之后,有些会安全感缺失,焦虑和嫉妒爆棚,然后主动跟我分手。然后每个人和我分手都是那种又爱又舍不得的痛苦不堪的感觉,看得我也好心疼。所以如果你想要比较长期的关系,这是一把双刃剑,请小心使用。

而跟人妻交往就比较有意思了,因为人妻有老公提供安全感,所以并不需要从黄毛身上追求安全感的感觉。所以有老婆有女朋友的黄毛,对于人妻只有附加的吸引力而没有缺点。人妻可以尽情地和黄毛享受刺激的迷恋感,回去就可以享受老公提供的安全感。所以经常看到人妻的出轨对象也是已婚男人,也是这个原因。郭菊阳刚开始和我约会的时候,我同时有未婚的女朋友,我还把女朋友的照片给郭菊阳看,她看完更兴奋了然后对我更加迷恋。没有对未婚女性的那种副作用。

前面说到多偶有副作用,但是更有趣的是,单偶的副作用更大。如果你认定了一个女人,不再考虑和其他女人接触的可能性,你在女人眼中就会完全丧失吸引力。女人会因为缺乏安全感,一直跟你确认你能一心一意的为她。比如会叫你官宣,叫你把你的手机屏保设成她的照片,不让你接触任何其他女性甚至和男性朋友出去约会。而一旦你真的满足了她的这些所有的安全感需求,围着她转,她就会本能地觉得你竟然弱到被他控制,不能按照自己本心去找多个女人,绝对是一个废物。于是,她就会对你产生一个新的负面感觉:``无聊''。她会觉得你``没刺激感'',``没新鲜感'',对你``没有感觉''。这时候,虽然你还是一心一意地对她好,她已经开始在外面追求新的基因提供者以获得``感觉''了。

这就是为什么,你看很多又帅、又有才华、又有钱的完美丈夫,老婆还是会出轨(现代)或者离婚(老一辈)。在中国有汪小菲、焦恩俊,在国外有巴菲特、贝佐斯、比尔盖茨,等等。无论男人多优秀,只要领证、保持安稳,女性就会没有新鲜感和刺激感,想要找新的男人给她带来新的刺激,即使客观上这个男人比她优秀得多。而之后的出轨离婚,也便自然而然了。

有一次,我在深圳湾万象城的至正潮菜宴请一个清华学弟,他当时博士毕业在腾讯做AI研发工作。他跟我说他之前好长一段时间都在接触各种女生,终于最近交了个女朋友还挺喜欢的,就没有继续寻找了。当时我看着他的眼睛,很严肃地跟他说``\textbf{放弃多偶倾向,就是丧失男性魅力的开始。}''他突然安静,沉思了很久,然后突然瞪大眼睛,对我露出了崇拜的眼神。大喊``太有哲理了'',差一点就认我做义父了。

而且,保持与多个女人接触,事实上可以增加你能找到女朋友的期望总量。这也是很好理解的。假设每个女人$i$,你泡到她的概率是$p_i$。那么你泡$1$到$n$一共$n$个女生,你期望得到的女朋友的数量就是$\sum_{i=1}^n p_i$. 很明显,你泡的女生越多,结果越好。如果你再把女人的价值$v_i$和泡她的成本$c_i$算进去,那么你泡$n$个女人期望收益就是
$$\sum_{i=1}^n \left(p_i v_i - c_i\right)$$

这个建模还可以继续细化,比如你的时间和金钱有限,所以你泡第一个女人和第十个女人的成本是不同的。所以你可以把成本分为时间和金钱,然后就可以把泡到一个女人$i$的概率变成一个关于在这个女人身上花的时间$t_i$和金钱$w_i$函数$p_i(t_i, w_i)$,然后把你要泡那些女人变成一个有时间和金钱总量限制的凸优化(convex optimization)问题。

但是这不重要,这些公式其实都告诉我们,你看到一个女人,只要泡她的概率乘以收益,大于泡她的成本,你就应该直接去上去泡,而不需要被所谓``专一''的女权社会规训所欺骗——这个社会女人从来就没有专一过。现代社会,很多人觉得男人同时泡几个女人就是``渣男'',缺德。而实际上,这不过是在做女人一直在做的事情罢了。很多喜欢把女人想象成圣女的``好男人''其实根本就不懂女人。女人看起来没有主动出击多个男人,但是女人只要通过发骚,比如在朋友圈或者小红书发自拍,就可以等着一群男人上钩,然后再在主动联系的男人里面挑那么五个十个条件好、有``感觉''的保持暧昧关系。几乎每个女人都会这么做,而且她们还可以装无辜说``我没有主动是他们主动联系我的'',把自己的责任撇得一干二净。而女权分子很喜欢给人洗脑的``我可以骚你不能扰'',更是要靠舆论的力量强迫社会消除她们的任何责任。

所以男性广撒网同时泡多个女性,无论是从自身利益出发还是从道德上的公平出发,都是毫无问题甚至是应该鼓励的事情。

\subsection{永远准备离开}

在和女人在一起的时候,要永远准备随时脱身。也就是说,永远不要害怕失去一个女人。这点和多偶有点像,但是比多偶还要更间接、更抽象。就是你多偶的时候,一个女人让你不开心,你就扭头找另外一个女人去了。但是实际上,女人只要看到她对你不好你扭头就走这一步,并不需要看到另外一个女人的存在,就会对你产生迷恋。一个优秀的男人,总是有自己的事情干,不管是自己的兴趣爱好、工作事业、还是别的女人。只有废物的男人才会整天围着女人转。而一旦女人感受到你这种扭头就走的魄力,就经常会产生迷恋。这种扭头就走不理的行为,有人称之为弃猫效应(abandoned cat effect)。

当年郭菊阳为了嫁有钱人把我甩了,我就觉得她傻逼,于是干自己的事情去了。结果到大二的时候我和正牌女友周游全国的时候,她还跑去清华想找我。这就是弃猫效应的应用了。

\begin{figure}[H]
\centering
\includegraphics[width=3in]{figures/guo_7.jpg}
\caption{郭菊阳在高中被弃猫之后写的}
\end{figure}

\begin{figure}[H]
\centering
\includegraphics[width=2.5in]{figures/guo_8.jpg}
\caption{弃猫效应持续到大二}
\end{figure}

最搞笑的是,很多女人似乎对自己会受弃猫效应有着清晰的认知,甚至以己度人,觉得对男人也有一样的效果。我曾经认识过几个女的,直接问我是不是她对我冷淡我就会喜欢她。我心里笑着想,她就算直接扑上来我都可能会嫌弃,她要是冷淡的话,我大概率一辈子都不会理她了。

\subsection{家庭暴力}

首先声明,无论是在中国还是西方国家,家庭暴力都是违法的。本章内容仅作为学术探究,请读者不要违反当地法律。

我有一个从海陆丰山区迁居到龙华山区的客家人小女朋友,每次见她,她都要求我用力的打她屁股。作为一个从小到大无论到哪里都遵纪守法的好(美国)公民,我第一次听到的时候,非常为难。我一开始还在想她是不是因为土客械斗失败的历史创伤造成了强迫性重复,作为潮汕土人的我还有点愧疚。但是在她的强烈要求之下,还是用力地拍打她的屁股,结果她脸和屁股一样变红了,超级兴奋。这时候我才知道打她是为她好。让乐于助人的我也变得特别开心和兴奋。

我经常能看到新闻,说有丈夫每天殴打妻子,社工苦苦劝告妻子离婚,结果妻子看丈夫还是眼神充满小星星,永远不离不弃,甚至会骂社工不怀好意破坏家庭,然后小孩还是一个一个生。这种新闻不是个例,甚至有研究表明,在家暴之后,女性怀孕的概率显著提高。

事实上女人天生会对有攻击性的男生产生好感,即使她就是被攻击的对象。有些女的会想,``他这么有攻击性,好有男人味,肯定能够好好保护我。''``他为什么打我不是打别人,肯定心里有我,把我当做他的女人,好有安全感。''

而其实相对于直接动手打女人,像我一样骂女人也是一种比较间接的合法版本的家暴。比如郭菊阳,高中的时候她为了嫁有钱人把我甩了开始,我就一直骂她贱货了。到这几年我每次操她的时候,也是打着屁股骂她是有钱就能上的贱货。也算是被我骂爽十几年了。

我也经常上网痛骂杨笠、痛骂张桂梅,很多人以为女人们尤其是女权分子讨厌死我了。其实我私下加过不少女权分子,她们很多面对我,都是那种战战兢兢,然后充满尊敬地跟我说话。其实你看前面我说的``女权分子本质上是发泄她们慕强和缺乏安全感的情绪''就懂了。当她们被一个强者指着鼻子痛骂,她们也就被骂爽了,骂舒服了,不想打拳了。要是我愿意操她们,明天就开始去商场挑小孩子衣服了,还打什么拳?

三年前有一个舔狗众多,父亲是公安局长的女群友。我约她去KTV,结果她一副跟我去玩是给我恩赐的表情,最后还爽约了。因为我曾经被网警拘留过,本身就看她这种公安二代很不爽。结果她还这么不尊重我,于是我就在群里骂了她两年。结果去年有一次,阴差阳错友好地请她吃了一次饭,她就疯狂地迷恋上我了。让我非常莫名其妙。或许这就是家庭(语言)暴力的魅力了。

\subsection{革命家和艺术家}

其实比起家暴她的女人,有一种女人更喜欢的有攻击性的性格,我称之为``革命家性格''。革命家不仅仅要有攻击性,还要有正义感和发自内心的勇气,敢于对抗所有的强权。而最看不起的,就是那些为了钱失去原则,什么人都能跪,什么生意都能做的小人。

女人会疯狂迷恋的另一种人,就是艺术家。像音乐、舞蹈,本身就是人类最原始的情感表达,比什么都直达人心。而美术与文学,更是能创造出不朽的作品,对抗时间的虚无,把人类情感凝聚成永恒。又有什么比艺术创作更伟大而值得献身的呢?而我,不也不吝耗费自己本来能用来赚几千万几亿的宝贵时间,凝聚成《人妻约会指南》这本艺术作品吗?

历史上的革命家和艺术家,从来不缺女人不要求金钱不要求名分,飞蛾扑火一般扑上去献身。即使在2025年的今天,走在纽约的街头,还有无数的白妹穿着切格瓦拉的T恤。唱出《怒放的生命》的汪峰,有着三个美貌的老婆和4个儿女。而唱《一无所有》的崔健,小女友也没间断过,几十年前就有女友未婚为他生下一个漂亮的女儿。

有时候我在想,革命家、艺术家,对个体来说,经常都是风险巨大且经常会穷困潦倒的职业。但是一个族群,如果没有革命家,没有艺术家,那么这个族群,还有什么伟大的可能,又有什么存在的意义?女性对革命家和艺术家的飞蛾扑火,何尝又不是一种族群至上的蜂巢思维呢?

我的父亲李松坚,作为一个低声下气跪着被双开的陈ら宇、蒋超ら赚黑钱的黑心裙带资本家(crony capitalist),被凌菲菲戴绿帽、背刺、霸占财产,真的是太合理了。又有哪个女人会真心喜欢这种丑陋猥琐的革命家的对立面呢?而凌菲菲拿到几十亿之后,也去追求她的真爱,原国家博物馆馆长,画家陈履生去了。

\subsection{保持主体性}

其实上面分析了这么多,其实吸引力法则总结起来就是一句话:\textbf{忘记女人,做你自己。}

钱总是流向不缺钱的人,而女人总是扑向那些不缺女人的人。你如果以想要女人甚至是想要操逼的心态来读上面的章节,你一定是学不像的,最后总是会透露出猥琐的气息。

而要活得吸引力,就是要保持主体性,把注意力放在自己身上:健身、学习、创作、改变世界。谁敢侵犯你,你就破口大骂。哪个女朋友让你不舒服了,你扭头就走。把人生去追求科学和艺术等永恒的东西。如果你发自内心的爱自己、做自己、追求自己的理想,那么你自然而然能有最强大的吸引力了。\footnote{可以参考后面的章目《超越人妻》。}

\section{如何吸引人妻}

人妻和单身的女人相比,其实更好吸引。首先,对于一个现代的单身女性,她们既要你能够给她们迷恋的感觉,也就是需要让她们觉得你是强者,又要你给她们一定的安全感,也就是需要你有钱适当供养她们。而对人妻来说,绝大部分对安全感的需求都可以由她们老公来满足,所以你只要让她们产生迷恋的感觉就行了。

其实很多时候,让女人产生迷恋的感觉并不难,其实比长期维持关系要简单。因为一旦结婚长期生活在一起,丈夫的很多优点,就被当作理所当然,而丈夫的缺点,就会被一直注意到。这时候,就会造成一种强烈的缺失感和需要补偿的感觉。郭菊阳当着我的面说过,她后悔大学的时候没有多跟几个男的上床,感觉吃亏了。类似的话,我也不止一次听不同的女人说过。每个结婚的女人都觉得亏了,因为没法多试试几个男人。所以你只要多聊几个人妻,如果你刚好满足了某个人妻在当下的缺失感,即使你整体远远不如苦主,人妻也会想跟你谈恋爱上床。

话说人妻们对老公挑刺,是可以极其离谱的。她们骂老公的话,你们一句话都别信。我在美国读研究生的时候每天陪我吃饭的女饭搭子,后来嫁到宇宙中心加州帕罗奥图(Palo Alto)。她整天给我发短信、微信、WhatsApp,说她老公不赚钱她生活很苦。后面我跟他老公聊天,才知道``老公不赚钱''是税前赚二三十万美金现金,而她一年要花税后五十万美金的意思。知道之后,我非常严肃地替她老公批评教育了她。结果只换来了一句``原来你也是恶臭抠门男''的评价。

但是读者们站在一个黄毛的角度,就知道这些有钱人的娇妻们是多么容易挑刺和不满,我们能找到的漏洞何其多!如果苦主年龄大,那么人妻就会想要找年轻的出轨;如果苦主长得丑,那么人妻就会找长得帅的出轨;如果苦主赚钱少,那么人妻就会找高收入黄毛出轨;如果苦主学历低,那么人妻就会找高学历的出轨;如果苦主在外面包二奶,那么人妻就会随便出轨;如果人妻就是苦主包的二奶,那也会随便出轨!

所以即使你是一个3分男人,你就同时撩那么100个7分男人的老婆,展示自己的优点。总会有时候,你的某个优点,刚好是某个人妻心理缺失的那一块,这时候她就会对你上头、迷恋,把你当做基因提供者了。

说实话,我泡郭菊阳的经历没什么好讲的,因为我没什么短板,说自己经历有点``数值怪以为自己可有操作了''的感觉。你想泡李松坚领证的好老婆许洁\footnote{见附录\hyperref[sister]{《我的姐姐李新莹》}}的话,也是没什么好讲的,毕竟李松坚楼盘烂尾被银行抽贷。一个长得像曹德旺的负资产的60多岁的中学学历的老头,操他老婆又有什么难的呢?

他老婆如果大家喜欢的话,大家闭着眼睛直接冲就是了。如果不喜欢的话,想到她应该靠长期做鸡最后还傍上大款上位成功赚了那么多钱,大家看在钱的份上也努力冲吧。许洁的电话我也不知道哪个,大家自己试试:13585833555、18019099644、18019243004、18101963394。

\begin{figure}[H]
\centering
\includegraphics[width=1.2in]{figures/xu.jpg}
\caption{``教授鸡''许洁在上海音乐学院官网的照片}
\end{figure}
