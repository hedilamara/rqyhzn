\chapter{超越人妻}

这本书的本意,是让读者们都能获得快乐且成为更优秀的自己。而与人妻约会,是现代社会中对大多数人活得快乐的最低成本方式,所以前面很多篇幅详细讲述和人妻约会的方方面面。但是在很多时候,我们和人妻约会,本质上也是为了我们自身的快乐。而快乐不一定需要在人妻之上获得。也可以找未婚的小妹妹,甚至很多人没意识到,快乐也不一定要从女人身上获得。而这一章,也是回到了书的本意,告诉大家和人妻约会之外,一个优秀的男人如何自我实现,并获得内心的快乐。

\section{避免沉船}

``沉船''是一个来自香港粤语的俗语,一开始指的是爱上妓女,后面又泛指爱上不应该爱上的人(炮友之类的)。

在和人妻约会的时候,沉船的可能性确实非常大。首先人妻因为有经验,更懂得怎么哄男人开心。其次又人妻因为从你身上要求的少很多,所以你会感到更多的温柔的感觉。而且人妻更有性经验,可能在床上会更开心。所以男人其实稍有不慎很容易在人妻身上沉船。而且一旦和人妻开始有爱情的感觉,那么人妻的安全感需求,包括对小孩的抚养责任,就会自然而然的放在你身上。如果人妻还是凌菲菲这种邪恶的人,可能会像凌菲菲吸李松坚一样,把所有的时间和金钱都吸干,给她和前夫的小孩。

所以在和人妻约会遇到沉船的时候,其实需要退后一步,重新审视这段关系。一开始你寻找人妻,不就只是因为成本低吗?现在因为人妻不要回报地对你好,你爱上了人妻。但是人妻能提供给你的也是有限的不是吗?陪伴、生育、养老,很多事情都无法提供。当你退后一步,把这些事想清楚了,你只是想要一个廉价的替代品,那么也就不会沉船了。

\section{回归少女}

前面说到,人妻有很多优点。比如容易泡,又比如一般不需要出钱负责。对于刚出社会的年轻中国男人,是性价比极高的一个选择。但是随着男性的个人进步与发展,一个高价值的人,是需要开始远离人妻的。

之前说过,如果结婚相当于买车的话,那么有租车、有网约车、有拼车。而约人妻就相当于直接走进隔壁邻居老王家把车开走。对大部分人来说,是性价比最高的开车方式。但是我现在还是没有去开邻居家的车,我买了一辆奔驰S级轿车,每天开车。为什么?因为十几万美金对我来说跟不用钱一样,所以我可以买一辆奔驰,什么时候想开就开,而且也不用怕脏。对我来说的一笔小钱,让我生活自由度和舒适度大幅提高。毕竟可能在郭菊阳老公读到这本书的性病防治部分以前,已经自己不注意患上了HPV之类的病了吧。公用的多脏啊。

而说到第一点想开就开,就要讲到当时21年我第一次回国的时候第一次联系上郭菊阳的事情了。当时我刚好因为准备要把卖房所得的几千万人民币换成美金投资美国科技股,于是我就来到了中国灰产之都深圳找人换钱。当时恰好郭菊阳在深圳,顺手发了个微信约她出来。一开始她约不出来,一下子说老公在家了,一下子说要带小孩了。因为我当时还有事需要回北京,没法在深圳呆太久,郭菊阳这么浪费我时间,我当场就发怒了,于是我就直接打了个30000元的微信转账过去,说``你拿去雇个保姆然后出来陪我。''于是她就出来了,虽然最后她也没收我的钱。

这说明什么?对我来说,我的时间成本很高。跟别人共享一个人妻,对我来说我损失的时间成本,比如如果在深圳待多一两天再去北京的成本,肯定就超过30000人民币了。也就是说,当你是一个年轻而收入普通的国男,你去找郭菊阳或者许洁自然非常省钱,毕竟有她们老公付生活费。\textbf{但是当你提高自己,让你的时间成本已经高到一定境界的时候,你就真正成长了。你已经超越了人妻,又可以回归少女了。}

在23年前9个月的时候,阿尔法星还是一个一人公司\footnote{其实一开始连公司都没有注册}。我一个人带着一台笔记本电脑全世界边旅游边创业,有时候在深圳约身份证4开头的妹妹们,有时候在上海约身份证3开头的妹妹们。当时我在上海,约了一个从福建客家山区去张江当初级程序员的小妹妹出去玩,开着李松坚的迈巴赫S680,从她在世纪大道的家里开车带她去上海迪士尼,拿着一大叠Priority Pass\footnote{Priority Pass是迪士尼乐园里面景点的收费插队的特殊门票}玩一整天,然后晚上再到外滩5号的茹丝葵牛排馆(Ruth's Chris Steak House)吃了一顿战斧牛排大餐。一整天下来,不算李松坚的迈巴赫,也就花了我5000人民币左右。从迪士尼的魔幻城堡到外滩的迷醉的霓虹灯下,看着福建妹妹在上海滩望着我迷醉的表情,这5000块,不比那30000块值钱很多倍?

\section{超越女性}

今年\footnote{2025年}四月,我有一天晚上在深圳南山区的某个五星级海景酒店和一个浙江女人约会完。事后那个女的就跟我作闹起来了,我跟她吵了几十分钟,实在受不了,于是就直接打车回福田了。当时在出租车上,看着购物公园和Coco Park酒吧街的灯光,恍惚之间,感觉那个\textbf{女人就像《聊斋》里面的女鬼一样,来到我身边,要吸取我的阳气。如果我让她把我阳气都吸光之后,我就死掉了。}

盯着酒吧的灯光的时候,我又突然回过神来。转念一想,我恍惚之间的幻想,其实难道只是幻想吗?我交往过这么多女性,没有一个女性在相处中不是对我进行索取,消耗。这种索取、消耗,本质上和女鬼吸取阳气又有什么区别呢?

只要和女性接近过,你肯定能感受到这种消耗。这种接近不一定是要有过性关系或者结婚,只要确定男女朋友,或者稍微暧昧一下,甚至你跟你周边女同学女同事稍微多聊几句,你都能感受到这种消耗。你在学校有没有经历过关系稍微亲近的女生叫你帮她写作业?你在工作的时候是不是帮女同事搬桶装水?你在微信加了一个女人\footnote{现在很多人用“女性”,甚至“女生”指代成年妇女。本书不会有这种语言腐败。}之后,是不是聊两句天,女的就要跟你要微信转账零花钱?之前的胖猫自杀事件,在全亚洲闹得沸沸扬扬,起因就是胖猫被一个重庆的捞女谭竹用要零花钱的方式掏空了所有金钱然后自杀。

而一旦关系升级到男女朋友甚至夫妻关系,这种索取就会升级。比如在中国非常流行的要求工资卡上交,男性只能偷偷藏“私房钱”。这在全球任何其他国家都是闻所未闻的。甚至每年春节联欢晚会都会有的私房钱梗,把这种敲骨吸髓式的剥削当作娱乐,实在是恶俗不堪。

除此之外,还有在中国农村很流行的彩礼和中国城市很流行的买房加名,也是现在中国最常见的索取。尤其是买房加名,很多人没意识到买房加名的严重性。在一二线城市,一套房子动辄几百万上千万,即使只算女人能分走的一半,也远远比全国最贵的江西省的彩礼要高。最近被判刑的苏享茂案的罪犯翟欣欣,在和苏享茂接触之前,就是靠离婚分得了一套北京的房产,进而用这套房产去欺骗苏享茂。而我的父亲李松坚,他找的小三凌菲菲,先是靠谋杀我姐姐和找黑社会逼我妈放弃财产离婚,再靠和我父亲离婚结婚骗取上海几十套房产,包括一套价值3亿以上的檀宫别墅。\footnote{详情见\hyperref[sister]{附录A.4《我的姐姐李新莹》}}

我听过很多北京、上海的朋友,喜欢歧视外地人,说什么“只有外地人才给彩礼,我们北京人/上海人没有彩礼这回事”。但是实际上,彩礼最多的江西,现在普遍的彩礼不超过一百万。而北京人上海人被骗房产的数额,动辄百万、千万记。最极端的凌菲菲,直接骗走我的父亲李松坚几十亿。一个陌生女人,以传统婚姻为诱饵,骗取北上广的男人领证,没有生小孩,过一段时间,就能把一个男人的所有积蓄掏空。而且这种掠夺比彩礼更加隐蔽。一般女人会用很多话术去包装,让那些被掠夺的城市人不知不觉,甚至还以为是真爱,觉得女的和他离婚是自己做得不好。下面是一些女人要房产加名的时候常见的话术,大家以后见到的时候可以注意分辨:
\begin{enumerate}
\item 你不帮我买房就是不爱我。
\item 我嫁给你们家,我需要一个保障。
\item 你不给我加名是不是防着我不信任我。如果这样那还不如不结婚算了。
\item 我们结婚了就是一家人了。写谁的名字不都一样。
\item 只要是你出资,加名的时候没有区别,离婚的时候我也分不走。
\end{enumerate}

前面四点实在太过可笑,没有必要浪费篇幅在这里反驳。但是最后一点还是很有迷惑性的。因为一般来说,女的还会配合一两个似是而非的案例和女权自媒体的传销号的宣传。但是中国并不是判例法国家,即使有一两个女方分不了房产的特殊案例,也代表不了你离婚的时候不会被分房产。房产加名,在全世界各个国家,基本就是默认就是赠予房产价值的50\%。如果贷款只有你一个名字而房产有女方的名字,默认就是女方有房产无贷款价值的50\%,你也有房产无贷款价值的50\%,但是房贷只属于你一个人。据我所知,在中国以外的地方也都是如此。如果女方拿出案例告诉你有离婚判决是按照婚前出资比例分配,那么只证明一件事:中国法院的离婚判决非常随意。落到你头上,可能法官往另外一个方向随意一下,你拿到的就比默认50\%的比例还要少了。

其实比起零花钱、彩礼、房产加名这种直接掠夺更可怕的是,共同生活之后,女性对男性的无穷尽的人身、精神、经济控制。

也是今年四月的时候,我一个好朋友和一个长得超级漂亮的女人约会。做爱完他想睡觉的时候,结果这个女人一定要他给个红包,不然就一直吵不给睡觉。我当时在深圳,打着跨洋电话和他说到这件事,不断感到后怕。其实只要睡一起,同个屋檐下。本质上,你的所有物理弱点就暴露了。比如这个女的,就可以通过不让你睡觉,逼你打钱。而在那种状况下,你根本没有太多反制措施。越是高价值的男性,你越经不起女人的这番作闹。在这种闹事博弈之中,算到最后,一个高价值\footnote{衡量一个人价值的一个很好的简单指标,就是一个人的单位时间经济产出。}的人能做出的最理智的行为,其实也就是花钱安抚了事。而无论过错在谁都理智花钱安抚,也是我到目前为止还没出现在(别人写的)PDF里面的原因。

说到这件事,不得不说,对公司的管理其实也是一回事。如果你有一个上升期的盈利公司,跟任何普通员工纠结五险一金和离职补偿都是低价值且不理智的行为。我创立的量化公司阿尔法星研究,入职过的员工也挺多了。到目前为止,我们所有辞退的员工,即使有着100\%的过错,我都会非常慷慨地给超过法律要求的离职补偿(severance package)。避免高价值的我本人和公司陷入离职员工的报复性破坏和官司之中。当然了,未来如果出现衍复投资的高亢之类的盗取公司商业机密离职开公司的事情,我自然也会和2 Sigma一样把对方送进监狱的。

为什么我这么做,因为一个员工进入了你的公司,相当于给予了这个员工对公司的巨大破坏力。而安抚员工,减少这种破坏的可能性,是不得不付出成本的:无论是离职包之类的安抚成本,还是把对方送进监狱的司法成本。毕竟要成为南山必胜客,也是要付出成本上供的。一样的,一个高价值男人,绝对不要随随便便让一个女人进入你的生活。一旦一个女人进入了你的生活,她们可以破坏你人生的方式太多了,而你不得不付出巨大的代价才能回归正常。尤其是在中国,司法体系对公司资方的保护力,绝对是远远强于男女关系中对高价值的一方(通常是男方)的保护。在这种文化制度下,我的公司尚且需要面试8轮,宁缺毋滥,筛选掉超过99\%的求职者\footnote{其实很多被拒的求职者,可以说是误杀。但是在现在的文化制度下,不得不极度谨慎。}。为何你们在现代社会竟然敢随随便便交女朋友甚至结婚呢?

前面说的不让睡觉,也算是一种比较低劣且极端的情况了。但是我遇到的很多女人,即使刚刚认识没多久,即使她还脚踏两只船,也会故意在你家留下头饰、丝袜之类的女性用品,对你之后的女性好友宣示主权。我记得2014年我刚到美国读研究生的时候,有一个深二代女留学生,她男朋友是一个深圳的潮汕人。然后邀请我去她家。她聊到她男朋友,然后跟我抱怨潮汕家庭小孩多,然后她男朋友不是长子没法继承太多家产,她很不爽。然后说着说着,她就开始喝红酒,然后就在我面前哭。说实话,我到今天还是莫名其妙。难道她觉得她的旧男朋友(苦主)和新男朋友(我)都是潮汕人就不算出轨?虽然她还继续跟她男朋友在一起,但是我出于好心,有几次我开车去纽约,还是让她坐副驾同路。然后她就故意在我的车上留下了女性用品,试图破坏我和别的女人的关系。最后她也成功了,造成了我和别的女人的争吵。

而其实在对于有价值的男性来说最大的危险在于,如果与女性共处,女性就会不断的通过作闹消耗你,消耗你的精力和金钱。这种消耗是女性本能的行为。如果你向女人的作闹妥协了,你就会被女人榨干所有的时间精力。然后赚不到钱,然后女人便会判定你是低价值的人,然后离开你找下一个——这也是女人的本能行为,无法控制的。女人这种无理由作闹,消耗男人精力的行为,只要有男人跟她接近就会触发。就像之前章节说的,郭菊阳作为以前背叛过我的二婚带男娃女,约出来都敢对我摆脸色作闹。

作为一个男人,最重要的,是想清楚自己想要什么,自己的梦想是什么。我在我的微信签名写到:短期目标100亿美金,中期目标众议院议长,长期目标殖民半人马座阿尔法星。即使你的目标没有如此宏大,其实也可以:比如想要给父母更好的生活,比如想要老婆孩子热炕头。或者你的目标就是低俗到就想要多操几个女人,多生几个小孩都行。甚至你有一些奇怪的癖好比如你要收集完12星座女朋友或者原潮阳县\footnote{大概是现在的汕头市潮阳区和潮南区}内每个镇的女朋友都行。每当你被女性控制,丧失自我的时候,想一下自己的目标和梦想是什么。真的是像李松坚一样帮凌菲菲养别人的小孩吗?还是跟李松坚一样把通过腐败到手的金钱进贡给做过鸡的许洁?

\textbf{每次想要被女性控制的时候,回去想一下你自己的梦想,就不会误入歧途了。}

\section{兄弟会}

这一章的内容,算是本文最深的内容。从1644神州陆沉至今近400年,中国的道德体系从未彻底拨乱反正。在现代的中国,全社会厚黑学大行其道,以欺骗和背刺为荣,以仁义礼智信为耻。所以我在本文讲的观点,很多中国的读者,应该是会反对甚至愤怒。笔者并不在意。

在现代中国社会,仁义礼智信被打为汉人和儒家的封建糟粕,而基督教的道德观又无法引进,再加上文革前后官方思想一片混乱,于是在一片精神空虚之中,几乎全民皈依了七八十年代美国好莱坞宣扬的舶来的“爱情神教”。尤其是女人和李松坚之类的小人,更是笃信爱情神教。我每个月至少和5个新认识的大龄剩女聊天,非常神奇,不管是中国还是美国,南方还是北方,她们好像脑子被memcpy\footnote{内存复制}似的说出一模一样的话:“我还没有遇到过有感觉的人。”而李松坚之类的小人更是离谱,为了所谓真爱,能故意让我姐姐得上肺结核差点死掉,然后再故意让我姐姐传染肺结核给我\footnote{详见本书附录收录文章\hyperref[sister]{《我的姐姐李新莹》}}。幸好我抵抗力强,最后没有被传染。也就是说,一旦成为爱情神教的教徒,无论男女,似乎仁义礼智信都能置之脑后,甚至杀子杀女也能接受。

而爱情,说到底,就是人类促进交配繁殖的神经系统和内分泌系统的综合反应罢了。猪猪狗狗也一样有。所以我现在看到那些爱上我的女的,就一点波澜也没有——谁会不喜欢一个高中的时候就是全美国计算机奥赛第一名金牌,白手起家身价几十亿,然后拿起笔当剑,对抗不公的社会,写下无数文章以重塑民族精神的人呢?我是女的我也流水。郭菊阳每次对我发情的心理,更像是在2009年卖飞英伟达的股市韭菜。

而现在,每个月都有好多扑上来找我的女人。在我眼中,她们像是拿着100块人民币要去买一个比特币,然后说她认识谁谁谁当年100块买了比特币了。她们还会跟我说我要相信爱情,然后爱她们,然后再领证把50\%的资产甚至100\%的资产奉献给她们,如果不这么做就是亵渎爱情。我每次听一个外貌完全不同的新的女人说出一字不差的话,都觉得特别可笑。

\textbf{其实人类真正伟大的情感,是多个不存在血缘关系的男性之间,能够有兄弟一般的情感,一起为了一个共同的目标而奋斗。}

如果观察历史,几乎所有伟大的事业,都是来源于多个男性,靠着高于纯粹利益的目的,形成一个共同体完成的。这种精神,在历朝历代都是所有人歌颂的对象。从桃园三结义,到白帝城托孤和《出师表》。从社会顶层的文人、进士们,到社会最底层的戏剧,都对那种忠义赞叹。从运城到成都到汕头,人们都在祭拜、悼念关公和诸葛丞相。可惜到如今,很多人看到白帝城托孤,第一反应是刘备在试探诸葛亮。人心不古,真是让人叹息。

这种男性之间的兄弟般的联盟,不仅仅是在中国传统儒家文化中被倡导。在西方也被社会正面歌颂且倡导的。我在美国多年,美国的很多文化,其实就是从小训练人与人之间的信任与合作关系。小孩子从小就参加足球、橄榄球等重视团队合作的项目。然后很多大学生,会加入一些学生自治组织的兄弟会(fraternity housing)。一群男大学生,自愿申请加入一个不由学校或者政府领导的学生组织。一群人像兄弟一样,在一个大房子里同吃同住。很多人一辈子的友谊就是在兄弟会里面建立起来的。

笔者在美国留学的时候,也曾在麻省理工学院(MIT)里面一个类似的兄弟会里居住过。每天住在一起,和兄弟们一起做饭、喝酒、打乒乓球。本章节的标题“兄弟会”,也是取自美国大学的兄弟会组织。

毕业后我也在美国的硅谷的科技公司工作过好长时间。硅谷有一种创业文化,就是一个有钱人,买了一个硅谷的大房子,然后让一群有能力、有理想、但是没有钱的年轻人住到他的房子里面创业。屋主不收房租,免费给吃给喝,甚至还会给创业者一些资源,只需要换取一小部分的新公司股权。我在硅谷有一个朋友,结婚前就把一群参加过USACO(美国计算机奥林匹克竞赛)的朋友都叫到他家的大房子住一起。现在这群同住过的人,好多都在硅谷或者创业或者投资,身价加起来已经不知道有多少亿美金了。而我的公司刚开始创业的时候,我在深圳虽然没有硅谷的五百万美金大别墅,但是也租了一个出租屋,几个男性朋友住在一起,吃着猪脚饭和肠粉创业。也算是把那套硅谷的创业文化因地制宜地移植了过来。

很多人低估了一个以信念和理想而不只是利益结合在一起的一群男人的战斗力。我在小公司和超大公司都呆过。那种几十万人上百万人的大型组织,每天的权力斗争、繁文缛节、腐败贪污、无意义内耗,大概率浪费了整个组织80\%以上的劳动力。其实只要很少的人,只要能减少这些内耗,就能爆发巨大的力量。

我本人其实经历过类似的组织,是我的高中汕头市金山中学的计算机奥林匹克竞赛班(金中竞赛班)。当我刚刚加入金中竞赛班的时候,金中历史上连一个进入省队的人都没有。虽然汕头市第一中学有过IOI银牌,但是汕头市金山中学一个都没有。高一进入金中竞赛班不久,我就成为事实上的金中竞赛班的领导者。这个领导者,不是学校任命的,而是金中竞赛班的成员们内部公认的。作为金中竞赛班成绩最好的一位,我很热心地教同级和比我小的学弟学妹们。

更重要的是,真正的领导,不仅仅是权力,更是一份责任。当时金中基本上什么教学资源都没有,我就用我当时还很一般的英语,跟美国计算机竞赛国家队的教练要美国队的内部训练资料分享给竞赛班的队友们。然后当时金山中学的校长李丽丽,对整个竞赛班充满敌意。只要哪个学生晚上去竞赛班时间比较长,她就会命令所有老师煽动协同所有学生去霸凌那个竞赛班的学生。当时在我高二的时候,在广东的所谓“小高考”\footnote{当年广东有一个考试叫做“学业水平测试”,只要及格就能顺利拿到毕业证。对于金山中学这种重点中学的学生来说都很简单。但是当时李丽丽和金山中学的老师们把这个考试叫做“小高考”,并且把它当作一个对学生进行服从性测试和压迫的理由。}之前,政教处几个人跑到竞赛班,抓了几个学生一起去政教处进行批斗。当时被抓过去的三四个学生,本来是站一排的,我直接向前迈出一步,直接跟政教处老师争论。我说这个考试,对我们的前途一点影响都没有。我们的时间我们自己分配,我们的前途属于自己,不用你们管。然后我直接对政教处的老师说,要不要打赌,看看我明天考试是不是全A。

当时金中竞赛班的氛围是非常友爱的,真的比现代很多亲兄弟还要更亲。当时我写的博客文章《飞厦forever》\footnote{原载我的博客:https://sinyalee.com/blog/?p=236}里面我想到我从飞厦中学到金山中学学弟们,我当时写道:``听说中山那边的OI水平都是由师兄带师弟上来的。而我呢,从高二很少上飞厦的竞赛班,也没能凭借自己的能力辅导他们,只能让Sivon他们来读我和炫圭艰深难懂的程序。所以我真的很内疚,觉得自己没有尽到做师兄的责任。''

这种友爱也是互相的,我在博客文章《For WH》\footnote{原载我的博客:https://sinyalee.com/blog/?p=383}里面写到,吴宏当时对我说:``你很有天赋,OI\footnote{计算机奥林匹克竞赛}可以进省队的,可是还是要努力。还有你书也会读好的。你的那个两年之誓,很让人震撼,也给了我很多感触。说真的,NOI\footnote{全国计算机奥林匹克竞赛}金牌曾经是我一个梦想,现在不可能了,我希望你可以。''

当到我高二终于能代表广东到全国比赛之后,接触到湖南、江苏、上海那些全中国顶级计算机竞赛名校的学生,才知道我们广东和汕头的这种氛围是多么难得。在那些(广东的)北方的学校里面,计算机竞赛班同学之间,互相仇恨,背刺是常态。如果一个学生有了一些自己拿到的学习资料,都会藏着掖着不敢让同学知道。当时高二第一次出省比赛,深刻地感受到了那些学生的人品和氛围,也是我把我的大学升学志愿定为第一麻省理工学院、第二香港科技大学、第三广州中山大学的原因之一\footnote{详见\hyperref[3faces]{附录A.3《我的父亲李松坚》}}。

这种兄弟般的团结和友爱之下,在我领导金中竞赛班的那一年,汕头金中从一个默默无闻的三线城市学校,成为广东排名第一的计算机竞赛强校。2009年的全国计算机奥林匹克竞赛,全国20个金牌,汕头就占了2个,占了全国的1/10。

而现在再重新回看,在我们那一届之后,我带起来的金中竞赛班,又出了月之暗面创始人杨植麟,“脑王”郑林楷等人。在我和我们那一届同学的争取和构建的风气之下,金中竞赛班继续下去,成为一个不断培养人才的``兄弟会''。

\begin{figure}[H]
\centering
\includegraphics[width=2.2in]{figures/gdoi.jpg}
\caption{我那一届的计算机奥赛广东省赛城市排名}
\end{figure}

\begin{figure}[H]
\centering
\includegraphics[width=2.5in]{figures/stoi.jpg}
\caption{广东省赛我们获得市第一名和学校第一名两个奖杯}
\end{figure}

这就是忠义和男性友谊的优势,在个体层面,可能一个体系里面的小人可以占尽便宜,但是如果一个群体能够通过信仰、忠义或者友谊团结起来,超越囚徒困境,这个群体的战斗力,是远远大于一个由小人组成的群体的。而一个由信仰和忠义确立的群体,坚决排除、惩罚小人,也是绝对正义的。

\section{舔狗与小人}

上一章从正面讲伟大的男性友谊和伟大的人是怎么样的。这一章,会从相反的角度讲一下,那些小人是怎么样的。

在孔子和儒家的理论中,有一对非常重要的概念:``君子''和``小人''。儒家对君子和小人的定义其实很清晰:君子喻于义,小人喻于利。也就是说,君子是讲道义的,而小人只讲利益。以这个标准看,今天的中国,遍地都是小人,君子越来越少了。

之前说的,能够为了理想情同兄弟的人,自然是属于君子。小人没有忠义,是不可能有这种男性间的友谊和同盟关系的。你想一下,你身边是不是有很多这种人,一旦发现你优秀,他们就嫉妒你,仇恨你,落井下石,想要把你拉下水?这些就是孔子所谓的小人了。尤其是我就读过的清华大学。无论是教授还是学生,绝对是小人浓度最高的一个地方。我在金中竞赛班,即使我是成绩最好的一个,大家对我的称呼都是``新野''(Sinya)或者``新野师兄''。但是在清华,所有人一见面都是``野神''、``野牛''地叫。不只是对我这样,人与人互相之间,就是充满了这种假情假意的恭维。而在恭维背后,是无尽的落井下石。比如我在清华姚班上学的时候,我曾经问过同学期末考时间、课程评分标准之类的问题,结果都有同学故意告诉我错误的期末考时间,故意告诉我错误的评分标准,甚至最终导致我有一门课程挂科。

清华大学这种可怕的氛围,我直到2022年才重新体会到。当时因为加入了城堡证券(Citadel Securities),搬到了芝加哥。这也是我除了在清华四年居住在北京之外,人生唯一一次在内陆城市居住。当时晚上7点多加班结束后,太阳已经下山,一个人步行在芝加哥的CBD洛普区(Chicago Loop)的街道上。路边都是游荡的无家可归的黑人,裤子提到屁股中间,露出半条沟子。这些人看到我,像鬣狗看到猎物一样,恶狠狠的盯着我的眼睛,大概率是在看我什么时候露出破绽,可以上前咬一口。我一方面不敢直视那些黑人,怕激怒他们,另一方面又不停用双目的余光注意那些人的一举一动,防止他们什么时候突然对我进行偷袭。而我在清华,感受是一模一样的。在那一刻,芝加哥那些游荡的黑人,比城堡证券的同事更像我的清华同学。

说到我在清华的经历,就不得不说清华大学两大糟粕文化之一的``女生节''\footnote{另一大糟粕文化是``清华特奖''。}。每年3月7日就是清华大学所谓的女生节。在这个节日,清华的男生会组织起来,所有男生进行一场舔狗表演。男生要拼命贬低自己的人格去单方面舔女生。谁舔得越卖力,谁就越光荣。

当时我们2010级姚班在女生节的时候,有几个舔狗头子,要求我们姚班25个左右的男生,每个男生必须出几百块,给5个女生每人买几千块的礼物。当时我严正反对了几句,但是还是懒得跟那些人吵架,最后还是出了几百块。当时大学的时候,我账户里有500万人民币的零花钱。而我知道一些姚班的同班同学,一个月零花钱不到1000人民币,我为他们发声,他们却一声不吭,从不到一千的零花钱里面掏出几百给比他们富裕得多的女同学们。(我跟女生很熟,我知道她们的家庭状况。)

清华的这些男生当舔狗,甚至是到了一种匪夷所思的地步。就是我如果对一个女的付出,肯定是想要娶这个女的或者上这个女的。是希望有回报的。但是清华这群舔狗,我观察他们,他们似乎是不追求正面回应,甚至是在期望负面回应的痛苦的。越是被女人拒绝,他们就越享受那种痛苦的爽感。

也就是说,清华学生这个群体,对女人付出一切的舔狗和那些背刺同学的小人,都尤其多。这其实并不是巧合。\textbf{每一个舔狗都是小人,没有例外。}一个人要成为舔狗,必然首先要放弃自身的尊严。一个连自身尊严都可以不要的人,也就是一个把利益放在一切之上的人。这些人,自然为了利益可以背信弃义。他们把女人当作物品、战利品,然后放弃尊严去获取,同时会仇视、敌对那些他们认为有竞争关系的同性,落井下石。这就是那些背刺同学的清华学生同时也是女生节的舔狗的原因。

我在《我的父亲李松坚》\footnote{《我的父亲李松坚》全文收录于\hyperref[3faces]{附录A.3}}一文中写的李松坚的三张脸,一张是``对女神奉献一切的脸'',还有一张是``对员工骄横跋扈''的脸。当时写的这两张脸,其实不只是李松坚,而是代表一整类人:每个为女神丧失尊严奉献一切的舔狗,都是会欺压、背刺身边男性的小人。

虽然这篇本书主要是打算写给男性阅读,但是这对女性也有很强的借鉴意义。对女性来说,一个男人一旦不请自来给你当舔狗,殷勤献得太过,这个男人大概率是要尽量避开的。一个小人给你当舔狗的唯一目的就是得到你,而对小人来说,放弃尊严得到之后自然是不珍惜的。这样说来,其实捞女狠狠爆舔狗金币,还是有其正义性的。